% --------------------------------------------------------------
% This is all preamble stuff that you don't have to worry about.
% Head down to where it says "Start here"
% --------------------------------------------------------------
 
\documentclass[12pt]{article}

\usepackage[margin=0.7in]{geometry} 
\usepackage{amsmath,amsthm,amssymb}
\usepackage{tikz}
\usepackage{multirow}
\usepackage{verbatim}
 
\newcommand{\N}{\mathbb{N}}
\newcommand{\Z}{\mathbb{Z}}
 
\newenvironment{question}[2][Question]{\begin{trivlist}
\item[\hskip \labelsep {\bfseries #1}\hskip \labelsep {\bfseries #2.}]}{\end{trivlist}}
\renewcommand*{\proofname}{Solution}
 
\begin{document}
 
% --------------------------------------------------------------
%                         Start here
% --------------------------------------------------------------
 
\title{Optimization Assignment 1}%replace X with the appropriate number
\author{Ashutosh Kumar\\ %replace with your name
MCS201402} %if necessary, replace with your course title
 
\maketitle
 
\begin{question}{1} %You can use theorem, exercise, problem, or question here.  Modify x.yz to be whatever number you are proving
Minimize : $(x_1 - 3x_2)$. Subject to:
\begin{equation}\label{q1:e1}
-x_1 + 2x_2 \leq 6
\end{equation} 
\begin{equation}\label{q1:e2}
x_1 + x_2 \leq 5 
\end{equation}
\begin{equation}\label{q1:e3}
x_1,x_2 \geq 0
\end{equation}
Also, write down the dual problem and get an optimal solution
\end{question}
 
\begin{question}{2}
Let 
\(
S = \{(x_1,x_2,x_3) \mid x_1 - 2x_2 \leq 3;\; 2x_2 + 3x_3 \geq 4\}
\)
\begin{enumerate}
\item Is it a convex set? 
\item Does it have a direction vector $d$?
\item Find an extreme point of $S$?
\end{enumerate}
\end{question}
 
\begin{question}{3}
Let \(
 A = \begin{bmatrix}
 -1 & 2 & 0 & 1 \\
 1 & 1 & 1 & 0
 \end{bmatrix}
 \).
 \\ Let S = \( \{(x_1,x_2,x_3) \mid -x_1 + 2x_2 +x_3 = 6;\; x_1+x_2 +x_4=5;\; x_1,x_2,x_3,x_4 \geq 0\} \)\\ \\
Write down all extreme solutions of $S$
\end{question}
 
% --------------------------------------------------------------
%     You don't have to mess with anything below this line.
% --------------------------------------------------------------
 
\end{document}